\documentclass{neu_handout}
\usepackage{url}
\usepackage{amssymb}
\usepackage{amsmath}
\usepackage{marvosym}
\usepackage{graphicx}
\usepackage[pdftex]{graphicx}
\usepackage{subfigure}
\graphicspath{ {images/} }
\everymath{\displaystyle}

% Professor/Course information
\title{Group Charter - UFOs}
\author{Abby, Emily, Lydia and Peter (The Conspiracists)}
\date{March 2018}
\course{CS 7295}{Information Visualization}

\begin{document}

\section*{1 Proposal}
The purpose of the team’s formation is for the final project of CS 7295 Information Visualization. We ourselves are the stakeholders, and have the intentions of putting this in Abby’s and Emily’s portfolios if approved.\\

The first (well) known UFO sighting was in 1947. While flying his small plane, Kenneth Arnold, saw a group of nine high-speed objects near Mount Rainier in Washington. He estimated the speed of the saucer-shaped objects as several thousand miles per hour. Since then, hundreds of thousands of UFO sightings have been submitted as possible proof of extraterrestrial evidence. We want to prove an interactive visualization to help others analyze this interesting dataset and formulate their own opinions.\\

The dataset that is used was scraped, geolocated, time standardized, and gives detailed descriptions of each sighting by Peter Davenport, the Director of the National UFO Reporting Center in Washington State. 

\section*{2 Goals}
We aim to create an effective interactive visualization in order to give UFO experts or the general public new insights. In popular culture, unidentified flying objects is referred to as a suspected alien spacecraft. The definition also encompasses any unexplained aerial phenomena. The difficult task, as credible UFOlogists will admit, is proving that a extraterrestrial spacecraft has visited planet earth.\\

We hope to provide interactive and interesting visualizations through the use of D3 to answer some of the following interests:\\

1. What areas of the country are most likely to have UFO sightings? \\
2. Are there any trends in UFO sightings over time? Do they tend to be clustered or seasonal? \\
3. Do clusters of UFO sightings correlate with landmarks, such as airports or government research centers?\\
4. What are the most common UFO descriptions?

\section*{3 Member Roles/Responsibilities}
Lydia: Document Coordinator and Information Manager \\
Abby: Team leader and data processing expert \\
Emily: Coding expert and project manager \\
Peter: Meeting Facilitator and Information Manager\\

When it comes to the development of the web application, we will look to work in 1 week sprints in order to chunk work into small pieces and work quickly.

\section*{4 Ground Rules}
We intend to mostly communicate through online chats and use a source code management service. Other than that we will meet once every week or two weeks to facilitate discussions. In case of dissenting views, we
will each provide arguments and try to reach a compromise or take a majority vote whenever it seems useful.

\section*{5 Potential Barriers and Coping Strategies}
One difficulty could come from other obligations of the team members that causes them to be absent or to not
be able to give time to the project for a certain period. We will discuss our plans for the semester and note such
potential periods for each member to the best of our knowledge so that we can be prepared to adjust the project’s progress around these periods. We also intend to give particular assignments and deadlines to each member so that they can adjust their work as they wish.



\end{document}
